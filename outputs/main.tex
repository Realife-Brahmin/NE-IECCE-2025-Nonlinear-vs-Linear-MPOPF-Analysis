\documentclass[conference]{IEEEtran} % default

\usepackage{cite} % default

\usepackage{amsmath,amssymb,amsfonts} % default

\usepackage{hyperref} % for hyperlinks obviously
\usepackage{cleveref}
% The following formats are used so that when I call \cref{label_1} or \cref{label_1,label_2} or \crefrange{label_1}{label_4}: 
    % 1. I never get eq. or eqs. before my bracketed numbers.
    % 2. The numbers are bracketed.

% For single equation references
\newcommand{\crefLsub}[1]{\cref{#1}L} % Custom reference command for "L" equations

\crefformat{equation}{(#2#1#3)}
\Crefformat{equation}{(#2#1#3)}
% For multiple equation references
\crefmultiformat{equation}
{(#2#1#3)} % First reference
{ and (#2#1#3)} % Middle references
{ and (#2#1#3)} % Last reference
{ and (#2#1#3)} % Last reference
% For a range of equation references (e.g., (1)-(3))
\crefrangeformat{equation}{(#3#1#4) to (#5#2#6)}

\usepackage{algorithmic} % default
\usepackage{graphicx} % default

\usepackage{float} % to use H

\usepackage[subpreambles=true]{standalone}
\usepackage{import}
\usepackage{subfiles} % for invoking figures from within a section vs by running the main file

% \usepackage{hyperref} % for hyperlinks obviously

\usepackage{subcaption} % for caption of figures with multiple plots

% \usepackage{lipsum} 
\usepackage{textcomp} % default
\usepackage{xcolor} % default

\usepackage{ulem} % for strikethrough-ing text

% Include custom environments
\makeatletter
\@ifundefined{counterL}{
    \newcounter{counterL}
}{}
\@ifundefined{counterNL}{
    \newcounter{counterNL}
}{}
\makeatother

% Standard alignL environment for standalone equations
\newenvironment{alignL}
    {\setcounter{counterL}{\value{equation}}%
        \setcounter{equation}{\thecounterL}%
        \renewcommand\theequation{\arabic{equation}L}%
        \align}
    {\endalign\setcounter{counterL}{\value{equation}}\setcounter{equation}{\thecounterL}}

% alignLsub environment for equations inside subequations (adds a, b, c suffix)
\newenvironment{alignLsub}
    {\renewcommand\theequation{\theparentequation L-\alph{equation}}%
        \align}
    {\endalign}

% Standard alignNL environment for standalone equations
\newenvironment{alignNL}
    {\setcounter{counterNL}{\value{equation}}%
        \setcounter{equation}{\thecounterNL}%
        \renewcommand\theequation{\arabic{equation}NL}%
        \align}
    {\endalign\setcounter{counterNL}{\value{equation}}\setcounter{equation}{\thecounterNL}}

% alignNLsub environment for equations inside subequations (adds a, b, c suffix)
\newenvironment{alignNLsub}
    {\renewcommand\theequation{\theparentequation NL \alph{equation}}%
        \align}
    {\endalign}

\title{Analyzing the Optimality Gap between Linear and Nonlinear Multi-Period Optimal Power Flow Models of Active Distribution Networks}

% Comparative Analysis of Linear vs. Nonlinear Multi-Period OPF Models for Active Distribution Systems

% Redefine the IEEEauthorrefmark command for using numbers as superscripts
\makeatletter
\newcommand{\mysup}[1]{\@fnsymbol{#1}}
\makeatother

\author{
    \IEEEauthorblockN{
        Aryan Ritwajeet Jha\mysup{1}, \textit{SIEEE},
        Subho Paul\mysup{2}, \textit{MIEEE},
        and Anamika Dubey\mysup{1}, \textit{SMIEEE}
        }
\IEEEauthorblockA{\IEEEauthorrefmark{1}\textit{School of Electrical Engineering \& Computer Science},
\textit{Washington State University},
Pullman, WA, USA\\
\IEEEauthorrefmark{2}\textit{Department of Electrical Engineering},
\textit{Indian Institute of Technology (BHU) Varanasi},
Varanasi, UP, India\\
\IEEEauthorrefmark{1}\{aryan.jha, anamika.dubey\}@wsu.edu, 
\IEEEauthorrefmark{2}\{subho.eee\}@itbhu.ac.in}

\thanks{%This work is supported  by the U.S. Department offfffff
%Energy’s Office of Energy Efficiency and Renewable Energy (EERE) under the Solar Energy Technologies Office Award Number DE-EE-0008774 (awarded to Sukumar Kamalasadan).
 Authors ack}\vspace{-7mm}}

% \thanks{This work is supported  by the U.S. Department of
% Energy’s Office of Energy Efficiency and Renewable Energy (EERE) under the Solar Energy Technologies Office Award Number DE-EE-0008774 (awarded to Sukumar Kamalasadan).
% S. Paul is with the  School of Electrical Engineering and Computer Science, Washington State University, Pullman, Washington 99163, USA (e-mail: subho.paul@wsu.edu). K. Murari is with the Electrical Engineering and Computer Science Department, University of Toledo, Toledo, Ohio 43606, USA (e-mail: krishna.murari@utoledo.edu).
% }

\begin{document}

\maketitle


\begin{abstract}
    Multi-period optimal power flow (MPOPF) frameworks are increasingly gaining attention due to the growing integration of battery-based distributed energy resources (DERs) in electricity distribution networks (EDNs). While MPOPF problems are typically formulated as non-convex programming (NCP) using non-convex branch power flow models, slow convergence and high computational requirements have spurred research into linear programming (LP)-based OPF approaches, which introduce an optimality gap. However, to the best of the authors' knowledge no current work has explicitly investigated how EDN size and DER penetration levels influence this gap. In this article, we develop multi-period OPF models for EDNs, comparing NCP- and LP-based solutions for networks of different scales—specifically, a 10-bus (small) and an IEEE-123-bus (medium) system—under varying DER deployments. We analyze the resulting optimality gap and degree of infeasiblity via OpenDSS simulations. Our findings indicate that while the optimality gap of LP-based methods grows with network size, these approaches still deliver near-optimal solutions and offer substantially faster convergence than their NCP counterparts.

    % The development of the multi-period optimal power flow framework is gaining interest nowadays because of the large penetration of battery associated distributed energy resources (DERs) into electricity distribution networks (EDNs). Typically, the OPF problem is modeled as a non-convex programming (NCP) framework by employing the non-convex branch power flow model of EDNs. However, the slow convergence and large computation time of the NCP motivate researchers to explore the possibility of developing the linear programming (LP) based OPF portfolio at the expense of the optimality gap in the derived solutions. To the best of the authors' knowledge, no state-of-the-art research has investigated the impact of EDN size on the optimality gap. In this context, this article aims to develop multi-period OPF models for EDNs and analyze their NCP- and LP-based solution processes across different EDN sizes, specifically 10-bus (small), IEEE-123 bus (medium), and 730-bus (large) test systems. The optimality gaps in the ACOPF solution and decision variables for each network are assessed by simulating the control decisions in the OpenDSS platform. The simulation results indicate that while the optimality gap due to LP increases with network size, it consistently delivers near-optimal solutions with fast convergence.
\end{abstract}

\begin{IEEEkeywords}
Battery energy storage systems, distribution system, optimal power flow, distributed energy resources.
\end{IEEEkeywords}

\import{../sections/intro/}{intro.tex}

\subfile{../sections/theory/theory.tex}
% \import{../sections/theory/}{theory.tex}

\import{../sections/simulation/}{simulation.tex}

% \import{../sections/results/}{results.tex}
\subfile{../sections/results/results.tex}
% \import{../sections/conclusion/}{conclusion.tex}

\subfile{../sections/conclusion/conclusion.tex}

% \cite{bfm01,Nazir2018Jun,Nazir2019Jun,ddp_sugar_01,Qian2014Jul}


\bibliographystyle{IEEEtran}
\bibliography{bibFile}

\end{document}
